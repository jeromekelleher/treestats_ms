\documentclass{article}
\usepackage{fullpage}
\usepackage{color}
\usepackage{amsmath,amssymb,amsthm}
\usepackage{natbib}

\bibliographystyle{plainnat}


\newcommand{\R}{\mathbb{R}}
\newcommand{\Q}{\mathbb{Q}}
\newcommand{\Z}{\mathbb{Z}}
\newcommand{\N}{\mathbb{N}}

\renewcommand{\P}{\mathbb{P}}
\newcommand{\E}{\mathbb{E}}
\newcommand{\var}{\mathop{\mbox{Var}}}
\newcommand{\cov}{\mathop{\mbox{cov}}}
%\newcommand{\det}{\mathop{\mbox{det}}}
\newcommand{\deg}{\mathop{\mbox{deg}}}
\newcommand{\supp}{\mathop{\mbox{supp}}}
\newcommand{\sgn}{\mathop{\mbox{sgn}}}

\newcommand{\bone}{\mathbf{1}}
\newcommand{\st}{\,\colon\,}

% These macros are borrowed from TAOCPMAC.tex
\newcommand{\slug}{\hbox{\kern1.5pt\vrule width2.5pt height6pt depth1.5pt\kern1.5pt}}
\def\xskip{\hskip 7pt plus 3pt minus 4pt}
\newdimen\algindent
\newif\ifitempar \itempartrue % normally true unless briefly set false
\def\algindentset#1{\setbox0\hbox{{\bf #1.\kern.25em}}\algindent=\wd0\relax}
\def\algbegin #1 #2{\algindentset{#21}\alg #1 #2} % when steps all have 1 digit
\def\aalgbegin #1 #2{\algindentset{#211}\alg #1 #2} % when 10 or more steps
\def\alg#1(#2). {\medbreak % Usage: \algbegin Algorithm A (algname). This...
  \noindent{\bf#1}({\it#2\/}).\xskip\ignorespaces}
\def\kalgstep#1.{\ifitempar\smallskip\noindent\else\itempartrue
   \hskip-\parindent\fi
   \hbox to\algindent{\bf\hfil #1.\kern.25em}%
   \hangindent=\algindent\hangafter=1\ignorespaces}

\newcommand{\algstep}[3]{\kalgstep #1 [#2] #3 }
\newenvironment{taocpalg}[3]{%
\vspace{1em}%
\algbegin Algorithm #1. ({#2}). #3 }
{\vspace{1em}}

\newcommand{\randomuniform}[0]{\mathcal{R}_U}
\newcommand{\randomdiscrete}[0]{\mathcal{R}_D}
\newcommand{\algref}[1]{#1}


\newcommand{\tskit}{{\texttt{tskit}}}
\newcommand{\branch}{\mathcal{B}} % branch stat
\newcommand{\site}{\mathcal{S}} % site stat
\newcommand{\node}{\mathcal{N}} % node stat

\newcommand{\nw}{{w}} % node weights
\newcommand{\iw}{{\tilde w}} % initial weights
\newcommand{\aw}{{\bar w}} % allele weights
\newcommand{\Tw}{{W}} % allele weights

\newcommand{\plr}[1]{{\color{blue} \it #1}}


\begin{document}

\begin{center}
    A general class of tree-based statistics,
    and how to compute them
\end{center}

\emph{Other title ideas:}

Efficient computation of a general class of DNA sequence statistics

Theory and computation of single-site statistics of DNA sequences


%%%%%%%%%%
\paragraph{Abstract}
Here we describe how to efficiently compute single-site statistics
from population genetic data using the succinct tree sequence
encoding for correlated genealogies.
The method is general enough to compute any currently-defined statistic,
and we use this general framework to suggest a number of new statistics.
\plr{say what method is: sthg about weighting fns}
This formulation also makes it easy to see what statistic of the underlying genealogical trees
each sequence-based statistic is estimating,
and, if underlying trees are also available,
the same methods can be used to quickly compute these quantities.


%%% OUTLINE
% 1. Framework and algorithms for computing general stats
%
%     * Framework:
%
%         - single-site stats are functions of the genotype vector
%         - in infinite sites, the genotype vector of a mutation is determined by the nodes below it
%         - ex: divergence as a sum over trees and branches
%         - ex: ancestry proportions
%         - general definition, using summary function of weights propagated up the tree
%             * recursion relating weights on nodes of two forests that differ by a branch
%
%     * Review of tree sequences (quick)
%
%     * Algorithms:
%
%         - weight propagation across tree differences
%         - count by node length ("Node Statistics")
%         - count by branch area ("Branch Statistics")
%         - count by site ("Site Statistics")
%         - note on outputs (by node, by site, in windows)
%
% 2. Examples: in sections, one of each type of output
%
%     - ancestry proportions
%     - list of standard summary statistics: divergence, Fst, f4, covariance
%         * performance for one of these
%     - PCA: Krylov method
%         * performance
%
% 3. Correspondence to tree statistics
%
%     - math: in infinite sites, neutral model, E[site] = branch
%     - plot from simulation showing branch stats are less noisy versions of the site stats
%     - show time profile of PC1 as computed from rare variants and common variants



%%%%%%%%%%%%%%%%%%%%%%%
\section*{Introduction}

\plr{need something different here}
Today's vast quantity of whole-genome sequence
makes it possible to confidently measure quantities
between \emph{individuals} that previously had to be roughly estimated
averaging over populations.
For instance, XXX estimated $F_{ST}$ between XX from allozyme sequence,
% JK: FIXME what's the slatkin ref here?
but now we know that $F_{ST}$ can be viewed as $1 - t_{within}/t_{total}$ \citep{slatkin_fst}
which can therefore be estimated precisely by $1 - \pi_{within}/\pi_{total}$.

% {computational challenges}
Computation is beginning to be a major problem:
whole-genome sequence on the scale of the UK Biobank,
for instance, might have 500K individuals genotyped at 10M loci (??),
resulting in a genotype matrix of $5 \times 10^{12}$ entries.
Computing a single summary from this matrix is still relatively quick,
once loaded into memory,
but for many applications we are now interested in computing a large number of statistics
(for instance, consider the roughly $10^{11}$ different possible pairwise divergences).
Indeed, many sorts of modern inference machinery, such as deep learning,
depend on being able to compute enough statistics that this becomes a serious bottleneck in practice.

The \emph{succinct tree sequence} (or tree sequence, for brevity) is a data
structure that efficiently encodes the genealogical trees describing how a set of
individuals are related to each other at each point along their genome. The
data structure was introduced in the context of coalescent
simulation~\citep{kelleher2016efficient}, and lead to scalability increases of
several orders of magnitude over existing methods. It was subsequently extended
and refined for forwards-time simulations~\citep{kelleher2018efficient,haller2018tree}
where similar efficiency gains were made. Recent
work~\citep{kelleher2018inferring} has shown that tree sequence algorithms can
also be used to massively increase the scalability of methods for inferring
genome-wide genealogies, and shown that it is feasible to infer trees for
millions of whole genome samples.

The key to the remarkable scalability and efficiency of tree sequence
algorithms is the way in which the shared structure in adjacent trees along
the genome is encoded. As one looks across the genome the genealogies change
because of the results of recombination in ancestors.
However, nearby trees tend to share a lot of common structure,
and single genealogical relationships (e.g., individual $x$ inherits from individual $y$)
are often shared across relatively long distances,
which manifest as shared \emph{edges} across many adjacent trees in the tree sequence.
This redundancy can be used
to both store genome sequence and the associated genealogical relationships extremely compactly,
as well as to compute statistics highly efficiently~\citet{kelleher2016efficient}.
This paper generalizes the latter strategy to a much broader class of statistics.

\plr{Review of tree shape stuff in phylogenetics}
a good review of theoretical properties \citet{semple2003phylogenetics}.

% summary
In this paper, we present a general framework for defining and efficiently computing
``single-site'' genetic statistics,
i.e., statistics of aligned genome sequence that can be expressed as averages over values computed
separately for each site.
Patterns of genetic polymorphism in a population
are determined by the population's history of Mendelian inheritance,
as summarized by the underlying genealogical trees,
and so it is natural (and helpful) to express these statistics
in terms of the structure of the trees themselves.
This perspective turns out to both suggest efficient algorithms
for computing these statistics on tree sequences,
and allows us to treat single-site genetic statistics in a more general framework
of summaries of tree shape and inheritance.
The methods are implemented in the python package \tskit.


%%%%%%%%%%%%%%%%%%%%%%%
\section*{Framework and algorithms}

In this paper,
we consider only so-called \emph{single site} statistics
(which we will define concretely shortly).
Extensions to statistics involving the pairwise joint distribution of genotypes across sites,
such as linkage disequlibrium,
are straightforward, and are planned for future work.
Haplotype-based statistics may require a different class of algorithms.


%%%%%%%%%%%%%%%%%%
\subsection*{Theoretical framework}

Suppose we have genotyped a collection of $N$ (homologous) chromosomes
at a sequence of $L$ loci along the genome.
Suppose first that we wish to compute an additive function of the allele frequency spectrum.
In other words, suppose that the frequency of allele $a$ at a given site is $p_i(a)$,
and that the allele frequency spectrum
-- i.e., the proportion of alleles whose frequency is $p$ in our sample -- is $f(p)$,
and for some function $h()$ we want to compute $\sum_p f(p) h(p)$.
Since this is a simple average over sites,
we can compute this, summing over the $L$ sites, as
$\sum_p f(p) h(p) = (1/L) \sum_{i=1}^L \sum_a h(p_i(a))$.
This definition applies to multiallelic sites by assuming that for each allele
we can only see how many samples carry that allele,
not how the remaining samples are partitioned between one or more other alleles.
(Note also that this does not differentiate ancestral and derived alleles,
and so is a function of the \emph{folded} allele frequency spectrum.)
If we know where each mutation has occurred on the genealogy,
we can easily find the frequency of the corresponding allele by counting how many samples
occur below that mutation in the tree (and removing those that subsequently mutate).

There are $2^n$ possible genotype patterns for $n$ haploid samples genotyped at a biallelic locus,
and so a general statistic of biallelic genotypes could be represented by a vector of length $2^n$.
A more convenient way to parameterize many functions of genotype patterns
uses three ingredients:
\begin{itemize}

    \item[(a)] A list of $K$ vectors of \emph{initial node weights}, $(\iw_1, \ldots, \iw_k)$,
        where $\iw_i(j)$ is the ``initial'' value of the $i^\text{th}$ weight assigned to node $j$.

    \item[(b)] The \emph{weight} $\nw$ of a node in a tree
        is a $k$-vector of weights is $\nw = (\nw_1, \ldots, \nw_k)$,
        and is equal to the sum of initial weights of all nodes below it in the tree
        (including itself).

    \item[(c)] A \emph{summary function} $F : \R^k \to \R$ so that
        $F(\nw)$ is the ``summary'' statistic we compute

\end{itemize}
Roughly speaking, we use these to compute a statistic
by first propagating the initial weights up the trees,
and then evaluate the summary function on the resulting vectors of weights
for each allele.

% Sum weights for each allele.
Concretely, to compute the statistic at a given site $j$,
for each allele $a$ seen at this site,
let $\aw_j(a)$, the weight associated with this allele,
be the sum of the initial weights over all nodes who inherit allele $a$.
If the site has only one mutation, which produced allele $a$
at node $i$, then $\aw_j(a) = \nw(i)$.
Then, the statistic computed at site $j$ is
\begin{align}
    \site(F, \iw)_j
    &=
    \sum_a \aw_j(a) ,
\end{align}
and a summary statistic across $L$ sites is
\begin{align}
    \bar \site(F, \iw)
    &=
    \frac{1}{L} \sum_{j=1}^L \sum_a \aw_j(a) .
\end{align}
We call this the \textbf{site statistic} associated with $\iw$ and $F$.

\plr{Figure: example of propagating weights and labeling nodes with weights,
    in the case that weights are number of samples you are ancestor of.
    Also, this is a node statistic.}

\plr{Figure: example of branch and site statistics}

This same framework can be used to compute other properties of interest.
For instance, suppse that we'd like to compute
what proportion of the genomes of the final generation were contributed by each ancestor.
In the same framework we can do ancestry proportions.
\plr{EXPLAIN, DEFINE NODE STATS}

\plr{Introduce branch stats.}
The \textbf{branch statistic} associated with the branch of length $t$
above node $n$ on a tree of length $\ell$,
with total weight $\Tw = \sum_j \iw(j)$
and node weights $\nw$,
is
\begin{align}
    \branch(F, \iw)_n
    &=
    \ell t ( F(\Tw - \nw(n)) + F(\nw(n)) ) .
\end{align}
This is the summary value that would be added to a site statistic
if a single mutation occurred on this branch:
we must apply the summary function
to both the weight below $n$ as well as the weight \emph{not} below $n$
to account for both derived and ancestral alleles.

\paragraph{Summary}
We have defined three types of statistic using the same weighting scheme:
\begin{enumerate}
    \item \textbf{Site statistics}
        -- summarizes weights below each allele at each polymorphic site.
    \item \textbf{Node statistics}
        -- summarizes total weight below each node.
    \item \textbf{Branch statistics}
        -- summarizes total weight below each branch, multiplied by branch length.
\end{enumerate}


\plr{TO-DO:}
\begin{itemize}

    \item How will polarization work?
        Right now, Site and Branch stats are polarized but Node statistics are not.
        (To make Node statistics unpolarized, we need to also add $F(\Tw - w)$
        to the weight of each node.)
        Perhaps this is OK -- Site and Branch stats should correspond to each other,
        but Node statistics are different, so could have different behavior.
        BUT, people will need to compute polarized Site and Branch stats;
        we need to make it easy for them somewhere.

    \item What about normalization?
        In the initial implementation, Node statistics were divided by the length
        for which the node was an ancestor to any sample, rather than just total sequence length
        To make these more analogous to the other stats, we think we want to remove this.
        The denominator for the original case is easily computable, by using
        $F(x) = 1$ if $x>0$ and $F(x)=0$ otherwise, with appropriate (positive) weights.

\end{itemize}


%%%%%%%%%%%%%%%%%%
\subsection*{Tree sequences}

Quick review of tree sequence terminology
and storage:
nodes,
edges,
sites,
mutations.


%%%%%%%%%%%%%%%%%%
\subsection*{Algorithms for fast computation}



%%%%%%%%%%%%%%%%%%
\section*{Case studies}
% AKA "examples"


%%%%%%%%%%%%%%%%%%
\subsection*{Ancestry proportions}


%%%%%%%%%%%%%%%%%%
\subsection*{Population genetics statistics}


%%%%%%%%%%%%%%%%%%
\subsection*{Genomic PCA}

Let $G$ denote the genotype matrix, so that $G_{ja} \in \{0,1\}$
is the genotype of the $a^\text{th}$ individual
at the $j^\text{th}$ site, with `1` denoting the derived state
(we assume here biallelic markers).
Also let $P_j = (1/n) \sum_a G{ja}$ denote the vector of allele frequencies.
Then the \emph{genetic covariance} between samples $a$ and $b$ is
\begin{align*}
    C_{ab}
        &= \frac{1}{L} \sum_j (G_{ja} - P_j) (G_{jb} - P_j) . %  \\
        % &= \frac{1}{L} \left( (G - P \bone^T)^T (G - P \bone^T) \right)_{ab} .
\end{align*}
This can be computed by letting
$\iw_a = \delta_a$ and $\iw_b = \delta_b$
(these weights mark nodes above $a$ and $b$ respectively)
and $\iw_t = \bone / n$
(this weight gives the proportion of samples below each node).
Then, with
$F(\nw_a, \nw_b, \nw_t) = (\nw_a - \nw_t) (\nw_b - \nw_t)$,
we can compute the covariance as
\begin{align*}
    C_{ab} = \frac{1}{2} \bar \site(F, (\iw_a, \iw_b, \iw_t)) .
\end{align*}
This follows, including the factor of two,
because the summary function applied to the ancestral allele
at a site with derived allele frequencies $p_a$, $p_b$, and $p_t$
in sample $a$, sample $b$, and total, respectively, is
$F(1 - p_a, 1 - p_b, 1 - p_t) = F(p_a, p_b, p_t)$.

Computing the full covariance matrix for a large number of samples may be infeasible.
However, we can efficiently find $y^T C z$ for arbitrary vectors $y$ and $z$.
This works because if we set the initial weights to $y$,
then each allele's weight is equal to the sum of the entries of $y$
corresponding to samples carrying that allele.
Then, if we define
$F_{yz}(u, v, t) = (u - (\sum_a y_a) t) (v - (\sum_b z_b) t)$,
then
\begin{align} \label{eqn:yCz}
    \frac{1}{2} \bar \site(F_{yz}, (y, z, \iw_t))
        &= \frac{1}{L} \sum_j (\sum_a y_a G_{ja} - (\sum_a y_a) P_j)
                    (\sum_b z_b G_{jb} - (\sum_b z_b) P_j) \\
        &= y^T C z .
\end{align}

\plr{Hmph: to actually compute $Cz$ we need one weight vector for each sample, again.
    TODO: look up if you can do Krylov just by computing $y^t C z$ for a reasonable number of vectors.
    We could compute $Cz$ quickly if we could propagate down the tree:
    we'd have $(Cz)_a$ equal to the sum of the weights at all nodes \emph{above} $a$. }

Where does the signal from an eigenvector come from?
Let $y$ be an eigenvector of $C$ with eigenvalue $\lambda$, normalized so that $\|y\|^2 = 1$.
Since $Cy = \lambda y$,
we can write $1 = \sum_a y_a^2 = (1/\lambda) y^T C y$.
However, equation \eqref{eqn:yCz} gives $y^T C y$ as a sum over \emph{sites},
and thus distributes the variance in $G$ associated with $y$
across the tree sequence.
(This is true because of the singular value decomposition of $G$.)



%%%%%%%%%%%%%%%%%%
\section*{Duality of site and branch statistics}

Under a neutral, infinite-sites model of mutation with constant mutation rate across time,
the expected number of mutations per branch is proportional to its length.
This implies an isormorphism between ``site'' and ``branch'' statistics defined above.


%%%%%%%%%%%%%%%%%%
\subsection*{The duality}



%%%%%%%%%%%%%%%%%%
\subsection*{Separating mutational from genealogical noise}


%%%%%%%%%%%%%%%%%%
\subsection*{Visualizing the time profile of PCA}



%%%%%%%%%%%%%%%%%%%%%%%
\section*{Discussion}


\paragraph{Diploid statistics}
The statistics we develop here are in essence \emph{haploid} statistics.
These tools can be used to compute a statistic about polyploid individuals,
but they do not scale to large numbers of individuals.
For instance, to compute the heterozygosity of a set of samples
-- this is $(1/N) \sum_{i=1}^N 2 p_i (1-p_i)$, where $p_i$ is the derived allele frequency
in the $i^\text{th}$ individual --
one would need to have an entry in the weight vector for \emph{each individual}.
The genotype of a diploid at a site with a single mutation that occurred on the branch above node $n$
is determined by where that node falls relative to the diploid's two haplotypes in the tree:
if $n$ is on the path from either haplotype to their MRCA, the individual is heterozygous;
if $n$ is on the path from the MRCA to the root, the individual is homozygous derived;
and the individual is homozygous ancestral otherwise.
This suggests the following generalization of our current scheme for diploids:
for a list $(I_1, \ldots, I_N)$ of pairs of nodes,
and two vectors $w^{(1)}$ and $w^{(2)}$ of weights of length $N$,
add weight $w^{(1)}_i$ to each node on the path between the two nodes of $I_i$,
and weight $w^{(2)}_i$ to each node on the path from the MRCA of $I_i$ to the root.
With $k$ such weighting schemes, we would then need the summary function $F$
to take $k \times 2$ values as input.
Efficiently updating these ``diploid'' weights on the nodes should be possible,
but is not a straightforward extension of our current method.




%%%%%%%%%%%%%%%%%%%%%%%
\subsection*{Acknowledgements}


\bibliography{references}

\newpage
\appendix

{\Large \plr{OLD STUFF}}

%%%%%%%%%%%%%%%%%%
\subsection*{Efficient computation}

First we describe how to define
and efficiently compute a quite general set of functions of a collection of sequences
using tools from tree sequences.
In the next section we will specialize these to a natural set of statistics.

\plr{Since the ``weighting function' can take negtive values, maybe ``weight'' is not a good term.  Instead?}

\paragraph{Weighted site averages}
Suppose that we have $k$ groups of samples, $A_1$, \ldots, $A_k$.
% with $n_1$, \ldots, $n_k$ individuals in each, respectively.
The \emph{weight} we compute for a given site is a sum of weights computed for each allele at that site,
and the weights are found as a function of the vector of numbers of individuals in each group inheriting that allele.
Concretely, we specify a \emph{weighting function} $w(x_1, \ldots, x_k)$,
and we say that the \emph{average site weight} of a set of $L$ sites is
\begin{align} \label{eqn:average_site_weight}
    \site_w := \frac{1}{L} \sum_{i=1}^L \sum_{s \in I_i} w(x_1(s), \ldots, x_k(s)) ,
\end{align}
where $I_i$ is the set of alleles seen at that site and
$x_j(s)$ is the number of individuals in $A_j$ that carry allele $s$.

Before we describe how to compute this,
we give the analogous topological quantity,
that is equal to the expected weighted site average
under a continuum-sites model.

\paragraph{Weighted branch length averages}
The \emph{average branch length weight}
is computed almost just as above:
\begin{align} \label{eqn:average_branch_weight}
    \branch_w :=  \frac{1}{L} \sum_{i=1}^L \sum_{e \in T_i} \ell(e) w(x_1(e), \ldots, x_k(e)) ,
\end{align}
except that now $T_i$ is the genealogical tree at site $i$,
the sum is over edges $e$ in the tree, $\ell(e)$ is the length of the edge $e$,
and $x_j(e)$ is the number of individuals in $A_j$ that fall below $e$.

\emph{Note:} somewhat more generally,
we could assign a set of weights to the individuals in each group,
and let $x_j$ denote the total weight of the individuals in $A_j$
inheriting from that allele or edge.
For simplicity we do not pursue this here.
\plr{this is a good use of the word ``weight'' (although these could also be negative)}


\paragraph{Computation}
The \emph{tree differences} encoded by a tree sequence allow us to efficiently update the quantities $x$ used above.
As it is more straightforward, we first describe the algorithm
to compute the branch length version.

\plr{Rework these into three algorithms: one to maintain the node weights $x$
    (basically already described as `tracked nodes' in msprime)
    and the other two that use this.}

\paragraph{Total branch length weight algorithm}
In a tree sequence, every branch in any tree is associated with its terminal \emph{node}.
We will keep updated an array $x$ so that for any node $e$,
the value $x[e][j]$ is the number of samples in $A_j$ below node $e$ in the current tree.
This will be zero for any node \emph{not} appearing in the current tree.
We also maintain $T$, the total weight of the current tree.
\begin{itemize}
    \item \emph{Initialize} all elements of $x$ to zero
        except that $x[a][j] = 1$ for each $a \in A_j$ and $1 \le j \le k$.
        Also set $S=0$ and $T=0$.
    \item Pop the next \emph{tree difference} off the stack: an interval with endpoints $\ell < r$,
        a set of edges $\mathcal{R}$ to remove, and a set of edges $\mathcal{A}$ to add.
    \item \emph{Remove} each edge (parent, child) in $\mathcal{R}$ from the tree,
        let $u$ = parent, and while $u != -1$,
        subtract $x[child]$ from $x[u]$ and set $u$ to be parent($u$).
    \item \emph{Add} each edge (parent, child) in $\mathcal{R}$ to the tree,
        let $u$ = parent, and while $u != -1$,
        add $x[child]$ to $x[u]$ and set $u$ to be parent($u$).
    \item For each node that has changed, let $w_\text{old}$ and $w_\text{new}$
        be the weights before and after the change in $x$,
        and add $w_\text{new} - w_text{old}$ to $T$.
    \item Add $T \times (r-\ell)$ to the running total, $S$.
    \item Return to (2).
\end{itemize}
\plr{should modify code to maintain a vector of $w$ values}
The average weighted branch length is obtained by dividing by the total number of sites.

The algorithm for sites is similar, also maintaining the array $x$,
and making use of the association of each mutation with a particular node in the tree.

\paragraph{Total site weight algorithm}
This works as the \emph{total branch length weight algorithm} except that steps (5--6) are replaced with:
\plr{gah this is terrible}
\begin{itemize}
    \item[5'] \emph{Find allele weights} for each mutated site in the tree:
        if $a$ is the ancestral allele,
        set $u[a]$ to be the value of $x$ at the root; and then,
        for each mutation at node $e$ with derived allele $b$
        and parent mutation occurring at node $p$ with derived allele $c$,
        add $x[e]$ to $u[b]$, and subtract $x[p]$ from $u[c]$.
    \item[6'] For each unique allele $a$ seen,
        add $w(v[a])$ to $T$.
\end{itemize}
\plr{check that algorithm deals correctly with more than one mut per node}

%%%%%%%%%%%%%%%%%%
\subsection*{General properties}

Here we describe some properties that we would like our statistics to have.

\paragraph{They do not depend on choice of reference allele.}
This is achieved by applying the \emph{same} weighting function to every allele.
Ancestral information can be incorporated by including an outgroup or ancestral sequence
as an additional group of size one.
Equivalently, we always work with \emph{unrooted} trees.

\paragraph{They do not depend on sample size.}
We'd like our statistics to be \emph{consistent estimators}
of a single quantity, and should therefore not depend on sample size in expectation.
\plr{insert this above}
This is achieved by always computing probabilies of sampling \emph{without replacement},
i.e., replacing each occurence of $p^k$ with $(x)_k/(n)_k$.
This lets us compare values of a given statistic
that are computed on different data sets with different sample sizes.

\paragraph{Equivalence of ``site'' and ``branch length'' statistics}
The definitions above are made so that for any weighting function,
$\site_w$ is a moment estimator of $\branch_w$, i.e.,
that $\E[\site_w] = \branch_w$ for \plr{somehow say continuum of sites here}.


\end{document}
